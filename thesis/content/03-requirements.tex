\chapter{Requirements}
\label{sec:requirements}
The general (\req{B}) requirements of this thesis will be summarized by two major points: 

\begin{itemize}
	\item[\req{B1}] \textbf{Performance:} Participating in the system should be possible with low-performance devices (such as smartphones). The overhead for the server on proxy decryption, attribute issuing and revocation should be reasonably low.  
	\item[\req{B2}] \textbf{Scalability:} The system should scale better than the current encryption scheme with respect to the number of file keys. This summarizes the performance impact of a growing number of clients.
\end{itemize}

In addition, the core (\textbf{\textit{C*}}) security requirements in the context of an \ac{MA-ABE} scheme are the following:
\begin{itemize}
\item[\req{C1}] \textbf{Collusion resistance:} For two users it should not be possible to combine their attributes to achieve a higher level of decryption power. Collusion resistance need also be ensured also on revocation. 
\item[\req{C2}] \textbf{Inter-Company Sharing:} Each company is only responsible for its own domain. This includes attribute and user administration, which translates to secret key generation and revocation. 
\item[\req{C3}] \textbf{Central Authority:} The Central Authority (\ac{CA}) shall not have global decryption power. At most an Attribute Authority (\ac{AA}) can decrypt user files of its own domain.  
\item[\req{C4}] \textbf{Secret Master key (if any):} Key recovery requires a secret and securely stored master key. It should solely function in the company domain and not globally. 
\item[\req{C5}] \textbf{Large Attribute and Key Universe:} The number of attributes and users shall not be restricted.
\item[\req{C6}] \textbf{Adding new Attribute Authorities:} It should be possible to add new AAs at runtime. Without either shutting down the system or recreating each key.
\item[\req{C7}] \textbf{Untrusted Attribute Authority:} A corrupt AA can only harm its own domain, but can not harm the outside system in any way. It cannot gain any additional information.
\item[\req{C8}] \textbf{Key and Attribute revocation:} Revocation is needed to handle user management in terms of attribute promotion, attribute demotion and key revocation. Forward secrecy should be provided.
\end{itemize}

\noindent Other (optional \textbf{\textit{O*}}) requirements are: 
\begin{itemize}
	\item[\req{O1}] \textbf{Traitor tracing:} A user in \ac{ABE} is described by his attribute set and is anonymous in this set. Misbehaving users, who sell their attribute keys to create a decryption black box, should be identifyable. \cite{liu2016practical}
	\item[\req{O2}] \textbf{Fine-grained access control:} The user shall not be bounded on defining fine-grained access policies which requires either an access tree \cite{bethencourt2007ciphertext} or an linear secret sharing scheme (\ac{LSSS}) \cite{yang2013dac} \cite{li2016secure} \cite{wu2017security} \cite{li2013matrix} \cite{liu2016practical}.
	%Some schemes restrict the user to threshhold access policies, where an user needs at least $n$ of $m$ attributes to encrypt the cipher text. \cite{chase2007multi} \cite{chase2009improving} Other approaches are restriced to $\ac{AND}$ gates which would translate to $m$ of $m$ threshhold gates. \cite{li2017two}
\end{itemize}