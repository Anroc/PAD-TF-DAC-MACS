\section{Thread Model}

TF-DAC-MACS describes six entities which will have different trust levels in our design. In the following the certificate authority and central server are merged, as well as the user and data owners are merged. 
\begin{itemize}
	\item The \textbf{Central Server} is assumed to try to break into the client communication. It might provide false information or wrong states to trick any identity to provide sensitive content. However, the central server will cooperate and will only deviate from the protocol if some advantage could be gained. Simply denying the cooperation – and so disrupting the service - does not belong to the goals of the central service. The central service does not cooperate with other malicious entities. 
	\item Each \textbf{Attribute Authority}’s goal is to access and eve drop content that is encrypted with attributes outside if its domain.  This distrust can be leveraged by a trust relation ship, where two attribute authorities can explicitly agree to trust each other. This mutual trust relation indicates that an trusted attribute authority is assumed to provide only truthfully information to the other trusted authority. Again it is assumed that an attribute authority does not cooperate with other malicious entities and that it will follow the protocol.
	\item \textbf{Cloud Storage Provider} are simply assumed to be honest-but-curious. They follow the protocol by try to decrypt content if possible. A cloud storage provider is separated from the other system and does not cooperate with any other malicious entity.  
	\item \textbf{Users} are untrusted. They try to collude with each other to achieve a higher level of decryption power. That means that they will exchange private attribute and two factor keys. However, it is assumed that they will not sell a decryption black box so that other external, unauthorized or revoked users use the black box to access sensitive content. 
\end{itemize}

This thread model is different from that proposed in TF-DAC-MACS. Here the central server and certificate authority are also semi-trusted. This is compensated by leveraging the trust level of the attribute authorities by introducing the trust-relationship. In TF-DAC-MACS a lot of trust is rooted in the certificate which was issued by the trusted certificate authority. In a real life system the trust of the certificate authority needs to be deescalated, since it is usually controlled by the same entity as the central server. To do so, for each certificate validation a second channel needs to be established: One to the issuer and one to the subject of the certificate, asking both for the validity of the certificate. 